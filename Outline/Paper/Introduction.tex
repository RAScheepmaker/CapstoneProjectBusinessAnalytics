\section{Introduction and Motivation} \label{sec:Introduction}
% Your outline should have the following structure: In the first section, provide a brief description of your topic, including some institutional information that will make possible an evaluation of the theoretical model you have chosen, to be presented in a later section. 

Credit cards are an important part of the American economy and culture.
According to the Consumer Financial Protection Bureau \citep{cfpb:2023}, consumers spent \$3.2 trillion on purchases using credit cards in 2022. 
Total credit card debt recently passed \$1 trillion, 82 percent of which is revolving (i.e. bearing interest, the remaining 18 percent is paid by the due date without being charged interest). 
The combination of income from interest, interchange fees and annual fees makes credit cards very profitable for the major banks, but they can also be profitable for a certain fraction of their users.
A key feature of many credit cards is their rewards structure, which returns a percentage of the user's spend back to the user in the form of cash back, reward points, or miles.%
\footnote{I consider points and miles to be the same for this project. Both can generally be redeemed for cash, statement credit, or used to book travel, either by transfering to loyalty programs of travel partners, or by booking flights and hotels directly through the bank's own travel portal.}
Credit card rewards programs are designed by banks to attract new customers and improve their loyalty, as well as to increase income through interchange fees and interest by stimulating (over)spending.
The CFPB estimates that the total dollar value of the rewards earned in 2022 exceeded \$40 billion, and the average rewards-earning account redeemed \$167.%
\footnote{According to Experian, the average American has 3.9 credit card accounts (\url{https://www.experian.com/blogs/ask-experian/average-number-of-credit-cards-a-person-has/}, accessed June 1, 2024).} 
These ``earnings'' are, of course, much less than the \$130 billion charged to consumers in interest and fees, making credit cards such a profitable enterprise for banks. 

As we will see in more detail in the next section, credit card rewards work partially through what is called the ``reverse Robin Hood'' mechanism, since to some extend it is the poor who, by paying interest and fees, subsidize the rewards of the rich \citep{wsj:2010}.
%\footnote{\url{https://www.wsj.com/articles/BL-REB-11033}}
More accurately, it is primarily the financially na\"{i}ve who are sponsoring the financially sophisticated, where the level of sophistication seems best measured by credit scores from the Fair Isaac Corporation (FICO), as opposed to income \citep*{agaretal:2023}.
%(Agarwal, Presbitero, Silva, and Wix, \citeyear{agaretal:2023}).
For those interested in travel using credit card points, or just saving some money through cashback rewards, it is therefore crucial to learn good financial habits and eventually become part of the financially sophisticated demographic with high FICO scores. 

Once credit card rewards are a net benefit, it might be worthwile to explore how we can optimize one's choice of credit cards to maximize this benefit. Which credit cards should different consumers select, and is there an optimal number before the marginal benefit of adding more cards becomes too small?  
These are the topics I would like to study in this project. 
My personal credit card portfolio has expanded from two to eleven credit cards over the last five years, increasing my net benefit from below two to about five percent of my entire annual spend.
With this project, I aim to turn this experience into an optimization algorithm that can be applied at scale to different types of consumers. This might result in some interesting insights, and perhaps an online tool, that could benefit many sophisticated credit card users who would like to stretch their budgets, travel more, or who simply enjoy optimizing their personal finances.
