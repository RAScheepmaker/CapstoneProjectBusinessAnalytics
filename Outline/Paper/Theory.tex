\section{Theoretical Model} \label{sec:Theory}
% In the third section, present your theoretical model, pointing out the implications of the model for the business problem you are trying to solve. 

Above we have seen that, in order to maximize the net benefits of credit card rewards, 
we should show ``sophisticated'' financial behavior.
Assuming we have such a sophisticated credit card user who never pays interest, and who spends her entire discretionary budget (staying within her means) on rewards credit cards, I will model the maximum net benefit for a different number of credit cards and for different types of users (e.g. users who will redeem credit card points for travel versus users who are only interested in cashback).

All reward credit cards come with certain point multipliers in specific categories, that will be multiplied with the spend in that category to calculate the number of points. 
For cashback rewards, the value of a point is per definition one cent, but for certain cards the value of a point can vary between a lower value (base value) and a higher value when redeemed for travel. 
Cards might also have an annual fee, or come with certain static benefits such as airport lounge access, travel or food credits, and discounts toward certain subscriptions such as Disney$^{+}$. 
In this project I will ignore sign-up bonuses (SUBs), which were 9.1 percent of total reward earnings in 2022 \citep{cfpb:2023}, since these high (but one-time) SUBs will always result in a very high marginal benefit for relatively little spend. Including them would  lead to the conclusion that it is always beneficial to open another account with a SUB into perpetuity.%
\footnote{People who keep opening new credit card accounts just for the SUB are known as \emph{churners}.}  
%SUBs are higher for below-prime scores, CFPB page 104).

The problem of optimizing a credit card portfolio can be considered a \emph{Dynamic Program}, since it requires an algorithm that uses simpler subproblems and stores intermediate solutions before the final solution is constructed. 
If we consider selecting $K$ credit cards from a dataset of $N$ possible cards, we start with a single card and add more cards later on, but only if this increases the net benefit. 
Therefore, the first subproblem is to the select the best single card and use it for all our spending categories. 
For adding an additional card, we have to iterate over all remaining possible cards and compare the value for each spending category to the previous card that we had already selected. 
If the total net benefit is higher, we keep the card in our portfolio and update the best card to use for each category. 
To select $K$ cards, we have to repeat the search through all $N$ possible cards $K$ times, making this an algorithm with complexity $\mathcal{O}(N)$.
%C categories, N card corpus size, K cards selected.
