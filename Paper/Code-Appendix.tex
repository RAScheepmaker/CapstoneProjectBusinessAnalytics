In this appendix, I present the \sR\ code of the two functions (\texttt{get\_buget()} and \texttt{get\_portfolio()}) that represent the core of this project, and that were needed by all the analysis presented in Sect.~\ref{sec:Results}.
Both functions are saved in the file \texttt{CCPortfolioFunctions.R}.

\singlespacing
\begin{lstlisting}[language=R]
get_budget <- function (income = "avg", budget_data) {
    # Given an income (before taxes), this function returns a named
    # vector with the spend per category.
    # "avg" is accepted as a special input argument to return the 
    # average budget from the Bureau of Labor Statistic's 
    # Consumer Expenditure Survey (2022), corresponding to an average 
    # income of $94,003. If a dollar amount is given as input instead, 
    # the function returns the average budget corresponding to the 
    # closest of 9 income bins.
        
    if (income == "avg") { 
      budget <- t(budget_data[2:19, 2])
    } else if (income < 15000) { 
      budget <- t(budget_data[2:19, 5] * income)
    } else if (income >= 15000 & income < 30000) { 
      budget <- t(budget_data[2:19, 7] * income)
    } else if (income >= 30000 & income < 40000) { 
      budget <- t(budget_data[2:19, 9] * income)
    } else if (income >= 40000 & income < 50000) { 
      budget <- t(budget_data[2:19, 11] * income)
    } else if (income >= 50000 & income < 70000) { 
      budget <- t(budget_data[2:19, 13] * income)
    } else if (income >= 70000 & income < 100000) { 
      budget <- t(budget_data[2:19, 15] * income)
    } else if (income >= 100000 & income < 150000) { 
      budget <- t(budget_data[2:19, 17] * income)
    } else if (income >= 150000 & income < 200000) { 
      budget <- t(budget_data[2:19, 19] * income)
    } else if (income >= 200000) { 
      budget <- t(budget_data[2:19, 21] * income)
    }
        
    budget <- as.numeric(gsub('[$,]', '', budget))
    names(budget) <- budget_data[2:19,1]
    return(budget)
}  
\end{lstlisting}

\newpage

\begin{lstlisting}[language=R]
get_portfolio <- function (K = 4, eta = 0, theta = 0, cards_data, 
    budget, verbose = TRUE) {
# Given K (number of cards), eta (fractional use of travel point value), 
# theta (fractional use of benefits), the cards data (CreditCards.csv
# in the form of a dataframe), and a budget (from get_budget()), this 
# function returns a named list, containing:
# - the optimal credit card portfolio
# - the total net benefit [$]
# - the marginal benefit per card
# - the return on spend after each additional card [%]
# - the card assignments for each spend category
# - the total spend [$]

##################################################
# Initialization
##################################################

total_spend <- sum(budget)
# Get the category names (are sorted by highest average spend)
categories <- names(budget)

# Specify benefits that should only be valued once (do not include
# 'benefit_credits', which will be valued per card and are assumed 
# to be stackable)
single_use_benefits <- c("benefit_globalentry", "benefit_lounge",
    "benefit_clear")
# Benefit values
ge_value <- 20      # Global Entry / TSA pre value in annual dollars
lounge_value <- 40  # Lounge Access in dollars per year
clear_value <- 189  # Clear Credit in dollars per year
single_use_benefits_values <- c(ge_value, lounge_value, clear_value)
benefit_value <- sweep(cards_data[ , single_use_benefits], 2, 
  single_use_benefits_values, '*')

# Initialize the category spend value matrix
cat_value <- matrix(0, nrow = length(cards_data[,1]), 
ncol = length(categories))

# The portfolio that contains the selected cards in order
cards <- c()
# The net and marginal benefit after adding each card to the portfolio
net_benefit <- c()
marginal_benefit <- c()
# Named vector that tracks which card to use for every category
card_assignments <- c()

##################################################
# Calculate Value of Spend per Card and Category (matrix)
##################################################
# K is the number of cards in the portfolio, while 
# N is the number of cards in the dataset.

for (n in 1:length(cards_data[,1])) {
for (c in 1:length(categories)) {
cat <- categories[c]
cap <- paste0(cat, "_cap")

# weighted average of point value, using eta
point_value <- eta * cards_data[n, "travel_value"] + (1 - eta) * 
                   cards_data[n, "base_value"]

if (cards_data[n, cap] == 0) {
# No spending cap, multiply spend * multiplier * value as usual:
cat_value[n, c] <- budget[[cat]] * cards_data[n, cat] * 
    point_value
} else if (budget[[cat]] <= cards_data[n, cap]) {
# There is a cap, but we stay below it:
cat_value[n, c] <- budget[[cat]] * cards_data[n, cat] * 
    point_value
} else {
# We spend more than the cap, so fall back to 1x multiplier 
# beyond the cap
cat_value[n, c] <- (cards_data[n, cap] * cards_data[n, cat] + 
       (budget[[cat]] - cards_data[n, cap]) ) * 
       point_value
}
}
}

##################################################
# Select the Optimal Portfolio
##################################################

for (k in 1:K) {
# Net benefit per card 
net_benefit_per_card <- rowSums(cat_value)  +
theta * (rowSums(benefit_value) + cards_data[, "benefit_credits"]) -
cards_data[, "fee"]

# Pick the card with max net_benefit_per_card:
max_ind <- which.max(net_benefit_per_card)
# Look up that card's name
cardname <- cards_data[ max_ind , "name"] 
# Add the selected card to the portfolio
cards <- append(cards, cardname )
# Store the cumulative net benefit and marginal benefit
if (k == 1) {
net_benefit <- append(net_benefit, net_benefit_per_card[ max_ind ])
} else {
net_benefit <- append(net_benefit, tail(net_benefit, n=1) + 
       net_benefit_per_card[ max_ind ])
}
marginal_benefit <- append(marginal_benefit, 
        net_benefit_per_card[ max_ind ])

# Assign spending categories with additional value to the card:
card_assignments[categories[cat_value[ max_ind ,] > 0] ] <- cardname

# Subtract the selected card's values from the value matrix 
cat_value <- sweep(cat_value, 2, cat_value[max_ind, ])
# And set negative categories to zero,
cat_value[cat_value < 0] <- 0
# Now we're left with positive categories that have remaining value 
# that we can retrieve with additional cards in the next iteration.

# Multiply all single-use benefits with (1 - single_use_benefits) 
# of max_ind, to set them to zero after the first time they 
# have been selected
benefit_value <- sweep(benefit_value, 2, 
    unname(unlist(1 - cards_data[max_ind, 
                  single_use_benefits])), "*")

# If the card can only be held once (Amex Platinum, Citi Custom 
# Cash, etc.) we need to make sure it will not be selected again 
# for just the benefits alone, or additional rewards in other 
# custom categories. This is accomplished by setting all the values 
# for the card, including benefits, to zero after selection. 
# We use ID to find duplicate cards that can only be held once.
# If cards can be held multiple times (e.g. BoA Customized Cash), 
# they will have different ID numbers, if not, the duplicates will 
# have the same ID.
same_ind <- which(cards_data[,"id"] == cards_data[max_ind ,"id"])
cards_data[same_ind, "benefit_credits"] <- 0
cat_value[same_ind, ] <- 0
benefit_value[same_ind, ] <- 0
}

if (verbose == TRUE) {
# Print output to the screen
print(sprintf("The optimal portfolio with a total benefit of 
$%.2f is:", tail(net_benefit, n=1))) 
print(cards)
print(sprintf("The marginal benefits are:"))
print(marginal_benefit)
print(sprintf("The total spend is: $%.0f", total_spend))
print(sprintf("The return on spend is: %.2f%%", 
100 * tail(net_benefit, n=1) / total_spend))
print(sprintf("Use the following card assignments:"))
print(card_assignments)
}

# Return a named list
return(list("cards" = cards, "net_benefit" = net_benefit, 
"marginal_benefit" = marginal_benefit,
"return_on_spend" = net_benefit / total_spend,
"card_assignments" = card_assignments,
"total_spend" = total_spend))
}

\end{lstlisting}
\doublespacing
