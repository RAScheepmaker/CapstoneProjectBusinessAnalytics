\section{Theoretical Model} \label{sec:Theory}
% In the third section, present your theoretical model, pointing out the implications of the model for the business problem you are trying to solve. 

Above we have seen that, in order to maximize the net benefits of credit card rewards, 
we should show ``sophisticated'' financial behavior.
Assuming we have such a sophisticated credit card user who never pays interest, and who spends her entire discretionary budget (staying within her means) using rewards credit cards, I will model the maximum net benefit for a different number of credit cards and for different types of users (e.g. users who will redeem credit card points for travel versus users who are only interested in cashback).

All reward credit cards come with certain point multipliers in specific categories, that will be multiplied with the spend in that category to calculate the number of points. 
For cashback rewards, the value of a point is per definition one cent, but for certain cards the value of a point can vary between a lower value (base value) and a higher value when redeemed for travel. 
Cards might also have an annual fee, or come with certain static benefits such as airport lounge access, travel or food credits, and discounts toward certain subscriptions such as Disney$^{+}$. 
In this project I will ignore sign-up bonuses (SUBs), which were 9.1 percent of total reward earnings in 2022 \citep{cfpb:2023}, since these high (but one-time) SUBs will always result in a very high marginal benefit for relatively little spend. Including them would  lead to the conclusion that it is always beneficial to open another account with a SUB, into perpetuity.%
\footnote{People who keep opening new credit card accounts just for the SUB are known as \emph{churners}.}  
%SUBs are higher for below-prime scores, CFPB page 104).

The problem of optimizing a credit card portfolio comes down to maximizing the net benefit by selecting, say at most, $K$ credit cards from a dataset of $N$ possible cards.
By itself, an algorithm to solve such a problem would, at best, need exponential-time resources, and is therefore NP-hard and intractable \citep*[][Sect.~6.2.3]{paargoly:2016}.
Luckily for us, people usually apply for credit cards one at a time, and over the course of months or years. 
We can therefore easily discipline our problem by assuming that people start with a single credit card, and subsequently add cards to their portfolio.
Smaller portfolios are also preferred for the sake of simplicity (and limited wallet space).
This additional structure turns our problem of selecting the optimal credit card portfolio into a tractable \emph{greedy} algorithm: the problem is broken down into smaller subproblems (``what is the best card to add now?''), and the best solution (highest additional value) is selected first.
It also uses elements from \emph{dynamic programming}, since we will need to store the intermediate solutions, and we will be using tables to calculate the marginal benefit that can still be extracted from the remaining cards.

The first subproblem is to the select the best single card, and use it for all our spending categories. 
For adding an additional card, we have to consider all remaining cards and compare the value for each spending category to the previous card that we had already selected.%
\footnote{In code, I implemented this by creating a matrix with the values per card and spending category. The values of the previously selected card are subtracted from this matrix, and negative values are set to zero. The remaining positive values show where additional value can be extracted from the corpus of cards.} 
The card with the highest additional net benefit (summing all categories and benefits, and subtracting fees) is added to the portfolio, and a table with the best card to use for each category is updated. 
To select $K$ cards, we have to repeat the search through all $N$ possible cards $K$ times, making this a feasible algorithm with complexity $\mathcal{O}(N)$.
%C categories, N card corpus size, K cards selected.
