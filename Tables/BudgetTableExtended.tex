\begin{table}[t!bh]
    \centering
    \begin{tabular}{ r c c l} 
        \hline
        Category & Expenditure [\$] & \% & Consumer Expenditure Survey Items \\ 
        \hline
        Everything else	& 7,786 & 20.18 & Household operations, Vehicle finance charges, Maintenance and repairs, \\
        & & & Vehicle insurance, Medical services and supplies, Reading, Tobacco, Miscellaneous \\
        Groceries & 6,362 & 16.49 & Food at home, Laundry and cleaning supplies, Other household products \\
        Dining & 4,222 & 10.94 & Food away from home, Alcoholic beverages \\
        Gas	& 3,120	& 8.09 & Gasoline, other fuels, and motor oil \\
        Utility	& 3,117	& 8.08 & Utilities, fuels and public services\\
        Home improvement & 2,606 & 6.76 & Household furnishings and equipment \\
        Online shopping	& 1,881	& 4.87 & 50\% Apparel and services, Pets, toys, hobbies, and playground equipment\\
        Drug store	& 1,481 & 3.84 & Drugs, Personal care products and services \\
        Travel (other) & 1,460 & 3.78 & 50\% Other lodging, 50\% Vehicle rental, leases, licenses and other charges, \\
        & & & 50\% Public and other transportation \\
        Phone & 1,431 & 3.71 & Telephone services\\
        Streaming & 1,020 & 2.64 & Audio and visual equipment and services\\
        Department store & 973 & 2.52 & 50\% Apparel and services \\
        Entertainment & 833 & 2.16 & Fees and admissions \\
        Cable internet & 698 & 1.81 & Other entertainment supplies, equipment, and services \\
        Hotel (portal) & 644 & 1.67 & 50\% Other lodging \\
        Airline (portal) & 423 & 1.10 & 50\% Public and other transportation\\
        Car rental (portal) & 394 & 1.02 & 50\% Vehicle rental, leases, licenses and other charges \\
        Office supplies & 128 &	0.33 & Postage and stationery \\
        \hline
        \hline
        Total & 38,576	& 100.00 & \\
    \end{tabular}
    \caption{The average credit card budget as derived from the 2022 BLS Consumer Expenditure Survey. The corresponding mean income (before taxes) is \$94,003, showing that, with this budget, 41 percent of the gross income can be spend on credit cards. The first column lists the 18 credit card categories that are used throughout this project, while the last column shows how the expenditure items were mapped to each of the credit card categories.}
    \label{tab:BudgetExtended}
\end{table}

